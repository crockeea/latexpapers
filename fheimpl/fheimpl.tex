\documentclass[oneside]{book}

%% common imports
\usepackage{microtype}
\usepackage{amsmath,amsfonts,amssymb,amsthm}
\usepackage{bm}
\usepackage{mathtools}
\usepackage{hyperref}
\usepackage[capitalise]{cleveref}
\usepackage[legalpaper, margin=1in]{geometry}
\usepackage{graphicx}
\usepackage{tabularx}
\usepackage{mathtools}
\usepackage{algorithm}
\usepackage[noend]{algpseudocode}
\usepackage{makecell}
\usepackage{minted}
\usepackage{caption}
\usepackage{float}
\usepackage{wrapfig}

% Make cleveref refer to appendices as appendices (not sections)
% https://github.com/latex3/hyperref/issues/362
\makeatletter
\AddToHook{cmd/appendix/before}{\crefalias{section}{appendix}}
\makeatother

\renewcommand{\th}{\ensuremath{^{\mathrm{th}}}}

% blackboard letters
\input{\CurrentFilePath/bbhead}
% left/right symbol pairs
\input{\CurrentFilePath/lrhead}
%\theorems, lemmas, collaries, etc
\input{\CurrentFilePath/thmhead}

% bibliography: *bibnames controls how many names get written in the bibliography before truncation to 'et. al', while *alphanames controls how many author names are used in the citation lable (e.g., Bae+22 vs BCD+22). See https://mirror.las.iastate.edu/tex-archive/macros/latex/contrib/biblatex/doc/biblatex.pdf
\usepackage[backend=biber,style=alphabetic,minbibnames=4,maxbibnames=6,minalphanames=3,maxalphanames=3]{biblatex}
% https://ctan.mirror.garr.it/mirrors/ctan/macros/latex-dev/base/ltfilehook-doc.pdf
\addbibresource{\CurrentFilePath/../refs/eprint.bib}
\addbibresource{\CurrentFilePath/../refs/other.bib}

% fixme: prefer \fxsetup{status=draft} in main docs instead of using `draft` here
\usepackage[multiuser,inline,nomargin,marginclue]{fixme}
\FXRegisterAuthor{e}{ee}{\color{green}Eric}
\FXRegisterAuthor{c}{cc}{\color{red}Craig}
\fxusetheme{color}

% show equation numbers only for referenced equations
%\mathtoolsset{showonlyrefs}

% custom operators/commands
\DeclareMathOperator*{\mdash}{\text{-}}
\DeclareMathOperator*{\argmin}{arg\,min}
\newcommand{\blockquote}[1]{\vspace{0.5em}\begin{tabularx}{\textwidth}{|X}#1\end{tabularx}\vspace{0.5em}}

% comment out to remove fixmes
\fxsetup{status=draft}

\usepackage{subfiles}

\newif\ifcompileasbook

\title{CKKS Implementation Notes}
\author{Eric Crockett}
\date{Last Updated \today}

\begin{document}
	\hypersetup{pageanchor=false}
	\compileasbooktrue
	\maketitle
	\hypersetup{pageanchor=true}
	\tableofcontents
	
	\chapter{Introduction}
	This article delves into CKKS bootstrapping. In CKKS, we must \emph{rescale} a ciphertext after\footnote{Or before, if using~\cite{cryptoeprint:2020/1118}.} each multiplication. Rescaling removes a modulus (or equivalently, increments the level of the ciphertext), hence, if the computation is deep enough, we eventually run out of moduli, and can do no more multiplications. Bootstrapping is a process which effectively ``resets'' the ciphertext to a lower level, thereby enabling addition multiplications. 

	\chapter{Basics}
%	\cite{Harvey_2014} talks about using $[0,2q)$ for efficiency in FFTs.
	\subfile{chapters/ckks}
	
	\chapter{Key Switching}
	\subfile{chapters/keyswitching}
	
	\chapter{Linear Transformations}
	\label{ch:lineartransforms}
	\subfile{chapters/lineartransforms}
	
	\chapter{Polynomials}
	\label{ch:polyeval}
	\subfile{chapters/polyapprox}
	\subfile{chapters/polyeval}
	
	\chapter{Bootstrapping}
	\label{ch:bootstrapping}
	\subfile{chapters/bootstrapping}

        \chapter{Matrix Multiplication}
        \subfile{chapters/matrixmul}
	
	\printbibliography 
	
	\appendix
	\renewcommand{\thechapter}{\Alph{chapter}}
	\chapter{KeySwitch Appendices}
	\subfile{chapters/keyswitching-sections/appendices}

        \chapter{FHE over the conjugate-invariant ring}
        \subfile{chapters/conjugateinvariant}

        \chapter{Modular Arithmetic}
        \subfile{chapters/montgomery}

\end{document}
