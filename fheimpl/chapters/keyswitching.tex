% !TeX root = ../fheimpl.tex
\documentclass[../fheimpl.tex]{subfiles}

\begin{document}
	\newif\ifksismain
	\ifcompileasbook
	\else
		\hypersetup{pageanchor=false}
		\title{Key Switching}
		\ksismaintrue
		\maketitle
		\listoffixmes
		\tableofcontents
		\compileasbooktrue
		\hypersetup{pageanchor=true}
	\fi

    \section{Introduction}
    This article delves into all operations that manipulate an FHE ciphertext modulus, and focuses on the algorithms implemented in Liberate-FHE. In particular, it covers basis extension, rescaling, and key-switching. \Cref{sec:notation} describes the notation, and \Cref{sec:intuition} gives the intuition for the fundamentals of key-switching. \Cref{sec:basisext,sec:rescaling} describe two key operations required for the actual key-switching algorithm, and \Cref{sec:ks} describes the actual key-switching procedure.
    
    \section{Notation}
    \label{sec:notation}
    We represent $x\in \Z_q$ by its \emph{canonical representative} $[x]_q \in [\ceil{-q/2}, \floor{q/2})\subset\Z$. We also use the \emph{standard representative} $\abs{x}_q \in [0, q)\subset\Z$.
    
    Most homomorphic encryption implementation utilize the Chinese Remainder Theorem (CRT), also called the Residue Number System (RNS) to avoid confusion with other uses of the CRT in lattice cryptography. Let $q_0, \ldots, q_{k-1}$ be co-prime moduli, and let $Q_j=\prod_{i=0}^j q_i$. Define $Q:=Q_{k-1}$ and $Q_{-1}:=1$. We denote $Q_i^* = Q/q_i\in \Z$ and $\tilde{Q}_i=(Q_i^*)^{-1}  \in \Z_{q_i}$. In CRT/RNS, we identify $x\bmod Q$ with $(x_0, \ldots, x_{k-1})$ where $x_i\in\Z_{q_i}$. We can reconstruct $x=\abs{x}_Q$ as $\sum_{i=0}^{k-1}\abs{x_i}_{q_i}\cdot \tilde{Q}_i\cdot Q_i^* \bmod Q$. We call $\set{q_0, q_1,\ldots, q_{k-1}}$ the \emph{RNS basis} of $x$. By ``an element $x$ in basis $\mathcal{B}$'', we mean $x$ in RNS form with respect to the moduli contained in $\mathcal{B}$. By the Chinese Remainder Theorem (CRT), there is precisely one integer $x$ such that $x$ in basis $\mathcal{B}$ is $(x_0, x_1, \ldots, x_{k-1})$. 
    
    \subfile{keyswitching-sections/intuition}
    \subfile{keyswitching-sections/basisconversion}
    \subfile{keyswitching-sections/rescaling}
    \subfile{keyswitching-sections/keyswitching}
    \subfile{keyswitching-sections/cornamiimpl}
    
	\ifksismain
		 \appendix
		 \section{Unification of Rescaling Notions}
\label{sec:rescaleextras}

Let $x\in \Z_{Q_\ell}$. \cref{sec:rescale-algs} gives three different definitions of rescaling $x$:
\begin{enumerate}
	\item $\mathrm{RS}(x)=\round{q_\ell^{-1}\cdot x} \bmod Q_{\ell-1}$ from~\cite{cryptoeprint:2016/421, cryptoeprint:2018/153, cryptoeprint:2018/1043, cryptoeprint:2020/1118}
	\item $\mathrm{RS}(x)=q_\ell^{-1}\cdot (x-x_\ell)\bmod Q_{\ell-1}$ from~\cite{cryptoeprint:2018/931, cryptoeprint:2019/688}
	\item $\mathrm{RS}(x) = \bracks*{\round*{\sum_{i=1}^k x_i\cdot\parens*{\tilde{q}_i\cdot\frac{t}{q_i}}}}_t$ from~\cref{sec:fastscaling}, which is adapted from~\cite{cryptoeprint:2018/117}
\end{enumerate}

In this section, we show that these three definitions are equivalent.

\subsection{Equivalence of (1) and (2)}	
\cite{cryptoeprint:2015/1134} describes why this process works. In short, we want to compute $\bracks*{\round*{\frac{1}{q_\ell}\cdot x}}_{Q_{\ell-1}}$ where $x\in \Z_{Q_\ell}$. Forget the final mod for a moment and just look at $\frac{1}{q_\ell}\cdot x$. Write $\frac{1}{q_\ell}\cdot x = \omega + \theta$ where $\omega\in\Z$ and $\theta \in [-1/2, 1/2)$. $x \equiv x_\ell\mod q_\ell$, so $\omega = \frac{x-x_\ell}{q_\ell}$ and $\theta = \frac{x_\ell}{q_\ell}$. Note that since the canonical representative $x_\ell=\bracks{x}_{q_\ell}$ is symmetric around 0, $\omega = \round*{\frac{1}{q_\ell}\cdot x}\in\Z$. Finally, 
\[\bracks*{\round*{\frac{1}{q_\ell}\cdot x}}_{Q_{\ell-1}} = \bracks*{\frac{x-x_\ell}{q_\ell}}_{Q_{\ell-q}} =\bracks{q_\ell^{-1}\cdot(x-x_\ell)}_{Q_{\ell-q}}\]

Now we extend this to $x = (x_1, x_2, \ldots, x_\ell)$ in RNS form:
\begin{align}
	\bracks*{\round*{\frac{1}{q_\ell}\cdot (x_1, x_2, \ldots, x_\ell)}}_{Q_{\ell-1}} = & \bracks*{\frac{(x_1-x_\ell, x_2-x_\ell, \ldots, x_\ell-x_\ell)}{q_\ell}}_{Q_{\ell-q}} \\
	=& (q_\ell^{-1}\cdot(x_1-x_\ell), q_\ell^{-1}\cdot(x_x-x_\ell), \ldots, q_\ell^{-1}\cdot(x_{\ell-1}-x_\ell))\in Z_{Q_{\ell-q}}
\end{align}	

\subsection{Equivalence of (2) and (3)}

In the context of rescaling, $q = Q_\ell$ and $t=Q_{\ell-1}$, so $t/q = 1 / q_\ell$. We rewrite the equation with these substitutions:

\[\mathrm{RS}(x) = \bracks*{\round*{\sum_{i=1}^\ell x_i\cdot\parens*{\tilde{q}_i\cdot\frac{Q_{\ell-1}}{q_i}}}}_{Q_{\ell-1}}\]

For simplicity, we will compute the output mod $Q_{\ell-1}$ in RNS form, so we really care about the rounded value mod $q_i$, $1\le i < \ell$. The $j^\text{th}$ component of $\mathrm{RS}(x)$ is
\[\bracks*{\round*{\sum_{i=1}^\ell x_i\cdot\parens*{\tilde{q}_i\cdot\frac{Q_{\ell-1}}{q_i}}}}_{q_j}\]

First, note that we can pull the first $\ell-1$ terms out of the rounding function because $q_i | Q_{\ell-1}$ for $1\le i \le \ell-1$. Thus we only need to compute the integral part of last term.

\begin{align}
	\round*{x_\ell\cdot\parens*{\tilde{q}_\ell\cdot\frac{Q_{\ell-1}}{q_\ell}}} = & x_\ell\cdot\parens{\tilde{q}_\ell\cdot Q_{\ell-1} -			 \bracks*{\tilde{q}_\ell\cdot Q_{\ell-1}}_{q_\ell}}/q_\ell \\
	= & x_\ell\cdot\parens{\tilde{q}_\ell\cdot Q_{\ell-1} -	1}/q_\ell
\end{align}

Therefore, the $j^\text{th}$ component of $\mathrm{RS}(x)$ is
\begin{align}
	\bracks*{\parens*{\sum_{i=1}^{\ell-1} x_i\cdot\tilde{q}_i\cdot\frac{Q_{\ell-1}}{q_i}} + x_\ell\cdot\parens{\tilde{q}_\ell\cdot Q_{\ell-1} - 1}/q_\ell}_{q_j} & = \bracks*{x_j\cdot\tilde{q}_j\cdot\frac{Q_{\ell-1}}{q_j} + x_\ell\cdot\parens{\tilde{q}_\ell\cdot Q_{\ell-1} - 1}/q_\ell}_{q_j} \\
	& = \bracks*{x_j\cdot (q_\ell^{-1} \bmod q_j) + x_\ell\cdot\parens{\tilde{q}_\ell\cdot Q_{\ell-1} - 1}/q_\ell}_{q_j} \\
	& = \bracks*{x_j\cdot (q_\ell^{-1} \bmod q_j) + x_\ell\cdot(-q_\ell^{-1} \bmod q_j)}_{q_j} \label{eqn:magic} \\
	& = \bracks*{(x_j-x_\ell)\cdot (q_\ell^{-1} \bmod q_j)}_{q_j}
\end{align}

where the \cref{eqn:magic} follows from the fact that $\tilde{q}_\ell\cdot Q_{\ell-1}$ is a multiple of $q_j$.

\subsection{Comparison}
We emphasize that~\cref{sec:fastscaling} is actually more general; in particular it works when $t \nmid q$.

Formulation 2 requires one modular addition and one (scalar) modular multiplication per output RNS component. If using standard representatives, there is an addition comparison and addition.

Assuming we precompute $\omega_{i,j}$ and $\theta_{i,j}$ for each output component $t_j$, Formulation 3 requires $\ell$ (scalar) modular multiplications and additions (ignoring the shared cost of computing $v$). Thus we should prefer formulation 2 when possible.

\section{Key Switching Notes}
\subsection{Generating Keys}
To generate a key-switch key for switching from key $s_1$ to key $s_2$:
\begin{enumerate}
	\item Generate $a^{(i)} \leftarrow U(R_{PQ})$ for $0\le i < \beta$
	\item Generate $e^{(i)} \leftarrow \chi$ for $0\le i < \beta$
	\item Compute $\mathrm{swk}^{(i)} = (b^{(i)}, a^{(i)})\in R^2_{PQ}$, where $b^{(i)} = -a^{(i)}\cdot s_2 + P\cdot[\hat{Q}_i]_{q_j}\cdot s_1 + e^{(i)}$ Note that when we view this in RNS form, the term $P\cdot[\hat{Q}_i]_{q_j}\cdot s_1 \cong 0 \bmod p_i$.
	%		where $b^{(i)}_j$ is the $j^\text{th}$ RNS component of $b^{(i)}$ and 
	%		\[b^{(i)}_j =\begin{cases}
		%			-a^{(i)}_j\cdot s_2 + [P]_{q_j}\cdot [\hat{Q}_i]_{q_j}\cdot s_1+e^{(i)}_j\bmod q_j & \text{$j$ corresponds to a normal prime} \\
		%			-a^{(i)}_j\cdot s_2 +e^{(i)}_j\bmod q_j & \text{$j$ corresponds to a special prime}
		%		\end{cases}\]
	Note: we can 
	\item Output $\set{\mathrm{swk}^{(i)}}_{0\le i < \beta}$, which we can view as a $\beta\times 2$ matrix $K$, where each element is in $R_{PQ}$.
\end{enumerate}

In the non-RNS version, key switch output $P\cdot s_1$ plus an encryption of zero; here we (roughly) output $P$ times a component of $\mathrm{RNS\mdash Power}_{\mathcal{C}'}(s_1)$, plus a (fresh) encryption of zero.

\subsection{Other Approaches}
\cite{cryptoeprint:2019/688} does not appear to be the final word on CKKS key switching; this section describes differences in later papers. \enote{TODO}

\subsubsection{Approximate Homomorphic Encryption with Reduced Approximation Error}
\cite{cryptoeprint:2020/1118} describes a different approach for $\mathrm{RNS\mdash Decomp}_\mathcal{C}$ and $\mathrm{RNS\mdash Power}_\mathcal{C}$ based on a base-$\omega$ decomposition. However, in section 2.3 it says that when using RNS, we should use RNS-digit-decomposition instead of base-$\omega$ decomposition. I confirmed that the decompositions are identical.

$\mathrm{KSGen}$ uses slightly different notation, but is identical.

		 \printbibliography
	\fi
\end{document}