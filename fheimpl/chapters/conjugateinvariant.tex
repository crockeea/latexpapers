\documentclass[../fheimpl.tex]{subfiles}

\title{Conjugate Invariant Approximate FHE}
\author{Eric Crockett}

\begin{document}
\ifcompileasbook
\else
\maketitle
\listoffixmes
\fi

\section{Mathematical Background}
A \emph{number field} is a finite extension of $\Q$. The cyclotomic number field is $K=\Q[\zeta_m]\cong\Q[X]/(\Phi_m(X))$, where $\zeta_m$ is an $m^\mathrm{th}$ root of unity. The degree of $K$ over $\Q$ is $\varphi(m)$. The \emph{ring of integers} of $K$ is $\mathcal{O}_K=\Z[\zeta_m]\cong \Z[X]/(\Phi_m(X))$. The degree of $\mathcal{O}_K$ over $\Z$ is $\varphi(m)$. Formally, $\mathcal{O}_K$ is the set of elements of $K$ which are a root of a monic polynomial with coefficients in $\Z$. These elements are called \emph{algebraic integers}.

An ideal $\mathfrak{a}\subseteq\mathcal{O}_K$ is an additive subgroup of $\mathcal{O}_K$ that is closed under multiplication by $\mathcal{O}_K$. Each ideal fo $\mathcal{O}_K$ is \emph{also} a free $\Z$-module of rank $N=\varphi(m)$, meaning it can be viewed as a lattice in $\R^N$ under the canonical embedding. RLWE uses the (principal) ideal of $\mathcal{O}_K$ generated by a modulus $1$.

There are two (related) lattices to consider. The first is the ``abstract ideal lattice'' defined by $\mathcal{O}_K$. The other is the Euclidean lattice defined by the canonical embedding of $\mathcal{O}_K$. Its basis is the embedding of the basis of the abstract ideal lattice. The ideal lattice problems are framed in terms of the Euclidean lattice, since this lattice has a notion of inner products and norms, which are required by the definition of SVP.

$\mathcal{O}_K$ is the ring actually used in FHE schemes, since it forms a full-rank lattice over $\Z$. Security is based on reductions from worst-case problems like Ideal-SVP or Ideal-SIS in the lattice formed by $\mathcal{O}_K$.\footnote{If we can solve RLWE over $\mathcal{O}_K$ on average, we can solve \emph{all} SIVP instances corresponding to ideals of $\mathcal{O}_K$. Some of these instances may be easy, but we assume there is at least one hard instance.} We can embed $\mathcal{O}_K$ into $\R^N$ via the canonical embedding.

\section{FHE Theory}
Traditionally, we rely on the cyclotomic number field and its ring of integers, which is well-studied, by which we mean that despite the expenditure of much effort, no one has been able to come up with an algorithm that solves SIVP on \emph{all} lattices induced by ideals of the cyclotomic integers.

\cite{cryptoeprint:2018/952} proposes a new FHE cryptosystem based on the ``maximal real subfield'' of $\Q(\zeta_m)$, defined by $K=\Q_{\xi}$, where $\xi=\zeta_m+\zeta_m^{-1}$. The ring of integers of this field is $R=\Z[\xi]\subseteq \Z[\zeta_m]$, which is called the conjugate-invariant ring. The paper shows that RLWE over $R$ is no easier than SIVP over ideal lattices in $O_K$. \enote{What is different about this proof than what is shown in~\cite{cryptoeprint:2017/258}?}

Note that $[K:\Q]=[R:\Z]=N/2$ (whereas for the fully cyclotomic ring, the degree is $N$). Therefore, the image of $R$ under the canonical embedding is $\R^{N/2}$. $R\cong \Z[X]/(X^{N/2}-1)$. Thus a vector of dimension $N/2$ can encrypt $N/2$ real plaintext slots.

\section{Security}
\cite{cryptoeprint:2018/952} argues that RLWE over the conjugate-invariant ring is secure by considering the attempts to break RLWE over the cycltomic ring: RLWE embeds into LWE in the form of matrices with specific structure (rather than arbitrary matrices used in standard LWE). For instance, over the cyclotomic ring, the random component of an RLWE sample corresponds to a circulant matrix in LWE. The authors argue that despite much effort, no one has been able to find an attack on RLWE that utilizes the special structure of the corresponding LWE matrix. As a result, the community has concluded that RLWE over the cyclotomic ring is as secure as LWE (with half the dimension). \enote{This may mean that to get the same depth, we have to use twice the ring dimension. For instance in standard CKKS, $N=2^{15}$ implies $q \le \mathcal{B}(N)$, and the plaintext space is $\C^{2^{14}}$. With this new scheme, the ring defined by $N=2^{15}$ yields a maximum modulus of $\mathcal{B}(2^{14})$, and yields $2^{14}$ real plaintext slots.}

In short, the conjugate-invariant scheme with $N=2n$ has the same security as CKKS with $N=n$.

\section{New FHE Scheme}

The plaintext space for CKKS with ring-dimension $N$ is $\C^{N/2}$.

Currently, FHE schemes rely on the hardness of RLWE over a cyclotomic ring. Until recently, the choice of the base ring was restricted because nothing was known about the hardness of (decisional) RLWE over non-cyclotomic rings.

Peikert et al.~\cite{cryptoeprint:2017/258} gave a (quantum) reduction from worst-case (ideal) lattice problems to the decision version of Ring-LWE, with no restrictions on the modulus or number field. \cite{cryptoeprint:2018/952} summarizes it this way: Peikert showed that the RLWE problem over the ring of integers of an arbitrary number field is no easier than SIVP over ideal lattices of the same number field.

\end{document}