\documentclass{article}

%% common imports
\usepackage{microtype}
\usepackage{amsmath,amsfonts,amssymb,amsthm}
\usepackage{bm}
\usepackage{mathtools}
\usepackage{hyperref}
\usepackage[capitalise]{cleveref}
\usepackage[legalpaper, margin=1in]{geometry}
\usepackage{graphicx}
\usepackage{tabularx}
\usepackage{mathtools}
\usepackage{algorithm}
\usepackage[noend]{algpseudocode}
\usepackage{makecell}
\usepackage{minted}
\usepackage{caption}
\usepackage{float}
\usepackage{wrapfig}

% Make cleveref refer to appendices as appendices (not sections)
% https://github.com/latex3/hyperref/issues/362
\makeatletter
\AddToHook{cmd/appendix/before}{\crefalias{section}{appendix}}
\makeatother

\renewcommand{\th}{\ensuremath{^{\mathrm{th}}}}

% blackboard letters
% blackboard symbols

\newcommand{\C}{\ensuremath{\mathbb{C}}}
\newcommand{\D}{\ensuremath{\mathbb{D}}}
\newcommand{\F}{\ensuremath{\mathbb{F}}}
\newcommand{\G}{\ensuremath{\mathbb{G}}}
% The standard definition of \H is that \H{o} will produce the letter ő (o with a double acute accent).
\renewcommand{\H}{\ensuremath{\mathbb{H}}}
\newcommand{\J}{\ensuremath{\mathbb{J}}}
\newcommand{\N}{\ensuremath{\mathbb{N}}}
\newcommand{\Q}{\ensuremath{\mathbb{Q}}}
\newcommand{\R}{\ensuremath{\mathbb{R}}}
\newcommand{\T}{\ensuremath{\mathbb{T}}}
\newcommand{\Z}{\ensuremath{\mathbb{Z}}}

\newcommand{\Zt}{\ensuremath{\Z_t}}
\newcommand{\Zp}{\ensuremath{\Z_p}}
\newcommand{\Zq}{\ensuremath{\Z_q}}
\newcommand{\ZN}{\ensuremath{\Z_N}}
\newcommand{\Zps}{\ensuremath{\Z_p^*}}
\newcommand{\ZNs}{\ensuremath{\Z_N^*}}
\newcommand{\JN}{\ensuremath{\J_N}}
\newcommand{\QRN}{\ensuremath{\mathbb{QR}_N}}

% left/right symbol pairs
% "left-right" pairs of symbols

%% NOTE: this requires \usepackage{mathtools} in the document preamble

% inner product
\DeclarePairedDelimiter\inner{\langle}{\rangle}
% absolute value
\DeclarePairedDelimiter\abs{\lvert}{\rvert}
% length
\DeclarePairedDelimiter\len{\lvert}{\rvert}
% a set
\DeclarePairedDelimiter\set{\{}{\}}
% parens
\DeclarePairedDelimiter\parens{(}{)}
% tuple, alias for parens
\DeclarePairedDelimiter\tuple{(}{)}
% square brackets
\DeclarePairedDelimiter\bracks{[}{]}
% rounding off
\DeclarePairedDelimiter\round{\lfloor}{\rceil}
% floor function
\DeclarePairedDelimiter\floor{\lfloor}{\rfloor}
% ceiling function
\DeclarePairedDelimiter\ceil{\lceil}{\rceil}
% length of some vector, element
\DeclarePairedDelimiter\length{\lVert}{\rVert}
% "lifting" of a residue class
\DeclarePairedDelimiter\lift{\llbracket}{\rrbracket}

%\theorems, lemmas, collaries, etc
%\theoremstyle{plain}            % following are "theorem" style
\newtheorem{theorem}{Theorem}[section]
\newtheorem{lemma}[theorem]{Lemma}
\newtheorem{corollary}[theorem]{Corollary}
\newtheorem{proposition}[theorem]{Proposition}
\newtheorem{claim}[theorem]{Claim}
\newtheorem{fact}[theorem]{Fact}
\newtheorem{techlemma}[theorem]{Technical Lemma}

% \newtheorem{invariant}[theorem]{Invariant}

%\theoremstyle{definition}       % following are def style
\newtheorem{definition}[theorem]{Definition}
\newtheorem{conjecture}[theorem]{Conjecture}
\newtheorem{construction}[theorem]{Construction}

%\theoremstyle{remark}           % following are remark style
\newtheorem{remark}[theorem]{Remark}
\newtheorem{example}[theorem]{Example}
\newtheorem{note}[theorem]{Note}

% equation numbering style
%\numberwithin{equation}{section}


% bibliography: *bibnames controls how many names get written in the bibliography before truncation to 'et. al', while *alphanames controls how many author names are used in the citation lable (e.g., Bae+22 vs BCD+22). See https://mirror.las.iastate.edu/tex-archive/macros/latex/contrib/biblatex/doc/biblatex.pdf
\usepackage[backend=biber,style=alphabetic,minbibnames=4,maxbibnames=6,minalphanames=3,maxalphanames=3]{biblatex}
% https://ctan.mirror.garr.it/mirrors/ctan/macros/latex-dev/base/ltfilehook-doc.pdf
\addbibresource{\CurrentFilePath/../refs/eprint.bib}
\addbibresource{\CurrentFilePath/../refs/other.bib}

% fixme: prefer \fxsetup{status=draft} in main docs instead of using `draft` here
\usepackage[multiuser,inline,nomargin,marginclue]{fixme}
\FXRegisterAuthor{e}{ee}{\color{green}Eric}
\FXRegisterAuthor{c}{cc}{\color{red}Craig}
\fxusetheme{color}

% show equation numbers only for referenced equations
%\mathtoolsset{showonlyrefs}

% custom operators/commands
\DeclareMathOperator*{\mdash}{\text{-}}
\DeclareMathOperator*{\argmin}{arg\,min}
\newcommand{\blockquote}[1]{\vspace{0.5em}\begin{tabularx}{\textwidth}{|X}#1\end{tabularx}\vspace{0.5em}}

% comment out to remove fixmes
\fxsetup{status=draft}

\begin{document}
	\title{Reducing Bootstrapping Cost by One Level}
	\author{Eric Crockett}
	\maketitle
	
	\listoffixmes
	
	\section{Introduction}
	CKKS homomorphic encryption requires \emph{rescaling} a ciphertext after performing a multiplication. This effectively reduces the scale factor on the message from $\Delta^2$ to $\Delta$, but ``costs'' a factor of $\Delta$ in the modulus (or equivalently, reduces the ciphertext \emph{level} from $\ell$ to $\ell-1$, or equivalently, removes on RNS modulus). You can only do this so many times before the modulus becomes too small to do further multiplications/rescaling (i.e., you reach level 0, or have only one RNS modulus remaining). Bootstrapping is a process to raise the modulus to a much larger value, enabling more multiplications and rescalings.
	
	\cite{cryptoeprint:2020/1118} introduces a proposal to reduce the approximation error in CKKS by rescaling \emph{before} multiplications rather than after.
	
	In the bootstrapping procedure, the first step (ModRaise) is to raise the modulus of the ciphertext to $Q_L$, the modulus of a fresh encryption. This produces a ciphertext with the correct modulus, but wrong message. The rest of the steps of bootstrapping restore the correct message. Raising the modulus doesn't require any ``secret'' information; the evaluator can do it without any assistance.
	
	The next step in bootstrapping is to apply a linear transformation. This step is known as \texttt{CoeffsToSlots}. There are many different techniques for achieving this step, but at least some of them involve immediately rescaling the output of the ModRaise step (e.g., to control noise).
	
	However, this is wasteful: for a given parameter set, a fresh encryption (and ModRaise) output a level $L$ ciphertext. In bootstrapping, however, we raise the ciphertext to level $L$, then immediately scale down to level $L-1$.
	
	We can avoid this by introducing a special ``bootstrapping modulus'' $p_b$. In the ModRaise step, we now raise the ciphertext modulus $Q_L\cdot p_b$. Since this step doesn't involve any secret information,
	\begin{itemize}
		\item the evaluator is free to choose $p_b$ on its own; this doesn't require any coordination with the encryptor
		\item this doesn't affect security.
	\end{itemize}
	
	Rescaling \emph{also} doesn't require any secret information, so the evaluator can rescale from level $L+1$ to level $L$ after ModRaise, and then start ModRaise at one level higher than before. As a result, the output of bootstrapping is also one level higher, which effectively reduces the depth of bootstrapping by one. The depth of bootstrapping is directly related to the efficiency of the overall scheme; we can now do one additional multiplication before bootstrapping again.
	
	\section{Security}
	Typically, adding additional moduli would reduce security because there also need to be key-switch keys for the same (larger) modulus. However, that only applies if the encryptor creates a ciphertext with more moduli. In this case, the evaluator is doing a ``public'' operation (i.e., one that doesn't require the secret key), so by definition this can't reduce security.
	
	\printbibliography 
	
	\appendix

\end{document}
