\documentclass{article}

%% common imports
\usepackage{microtype}
\usepackage{amsmath,amsfonts,amssymb,amsthm}
\usepackage{bm}
\usepackage{mathtools}
\usepackage{hyperref}
\usepackage[capitalise]{cleveref}
\usepackage[legalpaper, margin=1in]{geometry}
\usepackage{graphicx}
\usepackage{tabularx}
\usepackage{mathtools}
\usepackage{algorithm}
\usepackage[noend]{algpseudocode}
\usepackage{makecell}
\usepackage{minted}
\usepackage{caption}
\usepackage{float}
\usepackage{wrapfig}

% Make cleveref refer to appendices as appendices (not sections)
% https://github.com/latex3/hyperref/issues/362
\makeatletter
\AddToHook{cmd/appendix/before}{\crefalias{section}{appendix}}
\makeatother

\renewcommand{\th}{\ensuremath{^{\mathrm{th}}}}

% blackboard letters
% blackboard symbols

\newcommand{\C}{\ensuremath{\mathbb{C}}}
\newcommand{\D}{\ensuremath{\mathbb{D}}}
\newcommand{\F}{\ensuremath{\mathbb{F}}}
\newcommand{\G}{\ensuremath{\mathbb{G}}}
% The standard definition of \H is that \H{o} will produce the letter ő (o with a double acute accent).
\renewcommand{\H}{\ensuremath{\mathbb{H}}}
\newcommand{\J}{\ensuremath{\mathbb{J}}}
\newcommand{\N}{\ensuremath{\mathbb{N}}}
\newcommand{\Q}{\ensuremath{\mathbb{Q}}}
\newcommand{\R}{\ensuremath{\mathbb{R}}}
\newcommand{\T}{\ensuremath{\mathbb{T}}}
\newcommand{\Z}{\ensuremath{\mathbb{Z}}}

\newcommand{\Zt}{\ensuremath{\Z_t}}
\newcommand{\Zp}{\ensuremath{\Z_p}}
\newcommand{\Zq}{\ensuremath{\Z_q}}
\newcommand{\ZN}{\ensuremath{\Z_N}}
\newcommand{\Zps}{\ensuremath{\Z_p^*}}
\newcommand{\ZNs}{\ensuremath{\Z_N^*}}
\newcommand{\JN}{\ensuremath{\J_N}}
\newcommand{\QRN}{\ensuremath{\mathbb{QR}_N}}

% left/right symbol pairs
% "left-right" pairs of symbols

%% NOTE: this requires \usepackage{mathtools} in the document preamble

% inner product
\DeclarePairedDelimiter\inner{\langle}{\rangle}
% absolute value
\DeclarePairedDelimiter\abs{\lvert}{\rvert}
% length
\DeclarePairedDelimiter\len{\lvert}{\rvert}
% a set
\DeclarePairedDelimiter\set{\{}{\}}
% parens
\DeclarePairedDelimiter\parens{(}{)}
% tuple, alias for parens
\DeclarePairedDelimiter\tuple{(}{)}
% square brackets
\DeclarePairedDelimiter\bracks{[}{]}
% rounding off
\DeclarePairedDelimiter\round{\lfloor}{\rceil}
% floor function
\DeclarePairedDelimiter\floor{\lfloor}{\rfloor}
% ceiling function
\DeclarePairedDelimiter\ceil{\lceil}{\rceil}
% length of some vector, element
\DeclarePairedDelimiter\length{\lVert}{\rVert}
% "lifting" of a residue class
\DeclarePairedDelimiter\lift{\llbracket}{\rrbracket}

%\theorems, lemmas, collaries, etc
%\theoremstyle{plain}            % following are "theorem" style
\newtheorem{theorem}{Theorem}[section]
\newtheorem{lemma}[theorem]{Lemma}
\newtheorem{corollary}[theorem]{Corollary}
\newtheorem{proposition}[theorem]{Proposition}
\newtheorem{claim}[theorem]{Claim}
\newtheorem{fact}[theorem]{Fact}
\newtheorem{techlemma}[theorem]{Technical Lemma}

% \newtheorem{invariant}[theorem]{Invariant}

%\theoremstyle{definition}       % following are def style
\newtheorem{definition}[theorem]{Definition}
\newtheorem{conjecture}[theorem]{Conjecture}
\newtheorem{construction}[theorem]{Construction}

%\theoremstyle{remark}           % following are remark style
\newtheorem{remark}[theorem]{Remark}
\newtheorem{example}[theorem]{Example}
\newtheorem{note}[theorem]{Note}

% equation numbering style
%\numberwithin{equation}{section}


% bibliography: *bibnames controls how many names get written in the bibliography before truncation to 'et. al', while *alphanames controls how many author names are used in the citation lable (e.g., Bae+22 vs BCD+22). See https://mirror.las.iastate.edu/tex-archive/macros/latex/contrib/biblatex/doc/biblatex.pdf
\usepackage[backend=biber,style=alphabetic,minbibnames=4,maxbibnames=6,minalphanames=3,maxalphanames=3]{biblatex}
% https://ctan.mirror.garr.it/mirrors/ctan/macros/latex-dev/base/ltfilehook-doc.pdf
\addbibresource{\CurrentFilePath/../refs/eprint.bib}
\addbibresource{\CurrentFilePath/../refs/other.bib}

% fixme: prefer \fxsetup{status=draft} in main docs instead of using `draft` here
\usepackage[multiuser,inline,nomargin,marginclue]{fixme}
\FXRegisterAuthor{e}{ee}{\color{green}Eric}
\FXRegisterAuthor{c}{cc}{\color{red}Craig}
\fxusetheme{color}

% show equation numbers only for referenced equations
%\mathtoolsset{showonlyrefs}

% custom operators/commands
\DeclareMathOperator*{\mdash}{\text{-}}
\DeclareMathOperator*{\argmin}{arg\,min}
\newcommand{\blockquote}[1]{\vspace{0.5em}\begin{tabularx}{\textwidth}{|X}#1\end{tabularx}\vspace{0.5em}}

% comment out to remove fixmes
\fxsetup{status=draft}

\begin{document}
	\title{Cosine Similarity Benchmark}
	\author{Eric Crockett}
	\maketitle
	
	\listoffixmes
	
	\section{Introduction}
	This document describes Shai Halevi's algorithm for cosine similarity. We have a dataset with $N$ (key, value) records. Each key is an $\ell_2$-normalized $m$-dimensional vector. Let $A$ be the $m\times N$ matrix with all of these keys. We assume that each value fits in $k$ slots.
	
	There is a single secret query (row) vector $v$ of dimension $m$, also $\ell_2$-normalized.
	
	There is a threshold $\tau<1$, and a promise that no more than $M$ records in the dataset are $\tau$-close to $v$.
	
	The goal is to fetch the values corresponding to the keys that are $\tau$-close to $v$.
	
	The server has encryptions of the key and value for each record.
	Shai assumes we are using a ring dimension of size 65K, so there are 32K slots (note that $K$ is never a variable). The encrypted data is batched into blocks of size 32K records, where the keys for block $i$ are encoded into the slots of $m$ ciphertexts, and the values for block $i$ are encoded into the slots of $k$ additional ciphertexts. The number of blocks is $n=\ceil{N/32\mathrm{K}}$.
		
	The following list summarizes the notation:
	\begin{itemize}
		\item $N$ the number of records
		\item $m$ the number of slots per record key
		\item $k$ the number of slots per record value
		\item $n:=\ceil{N/32\mathrm{K}}$, the number of ciphertexts required to hold $N$ slots
		\item $\tau$ the dot product threshold. We return values for keys whose dot product with the query is $\ge\tau$.
		\item $M$ the maximum number of keys above the dot product.
		\item $C$ chunk size, used in the compaction step.
		\item $r:=\ceil{\log_{\frac{M\cdot C}{32\mathrm{K}}}(2^{-32})}$, the maximum number of 1s in a slice of size $C\cdot n$, with probability $2^{-32}$.
		\item $R:=\floor{32\mathrm{K}/C}$, the number of output ciphertexts
	\end{itemize}
	We require $C\cdot n\le 32\mathrm{K}$, and $r\cdot k\le 32\mathrm{K}$. \enote{Feel free to improve the variable names}
	
	The procedure proceeds as follows:
	\begin{enumerate}
		\item \textbf{Dot Products} Compute $u=v\cdot A$. We can isolate and duplicate each coefficient of $v$ using a multiplication by a (public) mask, followed by 16 shifts and 15 additions. Then we multiply each coefficient ciphertext by a packed row of $A$, and add the products together (for each block). This process has depth 2.\enote{I suspect Shai had a different algorithm in mind since he said it may take up to depth 8 for this step.} See \textrm{DotProds} in~\cref{alg:dotprod}.
		
		\item \textbf{Thresholding} Compare the slots of $u$ against the threshold $\tau$. Shai suggests using a polynomial approximation for this step, consuming 6 or 7 levels.
		
		\item \textbf{Compacting} The algorithm requires a promise that at most $M$ dot products will be larger than $\tau$, so the comparison vector has at most $M$ 1s in it. The output is $R$ ciphertexts, each with at most $r\cdot k$ non-zero left-justified slots. Roughly, the algorithm involves ``slicing'' the threshold vector and value matrix by taking $C$ slots from each of the $n$ ciphertexts (in the input, or in one row of the value matrix). We compact these slots into a single ciphertext, then compute a running sum. With high probability, the sum will be $\le r$, by design. For $i\in[1, r]$, we do an equality test on the running sum and multiply by the threshold vector to obtain an indicator vector with a one in the position of the $i\th$ one in the threshold vector. We can then extract the corresponding value coefficient from the value matrix, and compactly store it in an output ciphertext. See~\cref{alg:compaction} for details.
		
	\end{enumerate} 
	
	\begin{algorithm}
		\caption{Encrypted vector/matrix multiplication. $v$ is a single CT with data in first $m$ non-zero slots. $A$ is a matrix with $m$ rows, where each row is composed of $n$ ciphertexts. The $j\th$ ciphertext in the $i\th$ row is $A_{i,j}$.}\label{alg:dotprod}
		\begin{algorithmic}[1]
			\Procedure{Replicate}{$x$}
			\Comment $x$ has a value in one slot and 0s elsewhere
			\State \Comment Output has this value in all 32K slots
			\For{$i$ in range$(15)$}
			\Comment Hard-coded for 32K slots
			  \State $x \gets x + (x \gg 2^i)$
			\EndFor
			\State \Return $x$
			\EndProcedure
			\State
			
			\Procedure{ExtractCoeff}{$v$, $i$}
			\Comment Return a CT with $v_i$ in all slots
			\State $mask\gets \mathrm{Indicator}(i)$
			\State $w\gets mask \cdot v_0$
			\State \Return $\mathrm{Replicate}(w)$
			\EndProcedure
			\State
			
			\Procedure{DotProds}{$v$, $A$}
			\State prods$ \gets [0]*n$
			\For{$j$ in range$(m)$}
			  \State $v_i\gets \mathrm{ExtractCoeff}(v, j)$
			  \For{$i$ in range$(n)$}
			    \State prods[$i$]$ \gets \mathrm{prods[}i\mathrm{]} + v_i\cdot A_{i, j}$
			  \EndFor
			\EndFor			
			\State \Return prods
			\EndProcedure
		\end{algorithmic}
	\end{algorithm}
	
	\begin{algorithm}
		\caption{Compaction. $us$ is the output of the thresholding step, a vector with $n$ ciphertexts. $ps$ is a $k\times n$ matrix where each row $ps_i$ is $n$ packed ciphertexts. $ps_{i, j}$ means the $j\th$ ciphertext in the $i\th$ row}\label{alg:compaction}
		\begin{algorithmic}[1]
			\Procedure{RunningSum}{$v$}
			\State $rs\gets v \cdot \mathrm{Indicator}(0)$
			\State $acc\gets v$
			\For{$i$ in range$(1, 32\mathrm{K})$}
			\Comment Extracts the $i\th$ component of the running sum
			  \State $acc \gets v + (acc \gg 1)$
			  \State $rs \gets rs + (acc \cdot \mathrm{Indicator}(i))$
			\EndFor
			\State \Return $rs$
			\EndProcedure
			\State
			
			\Procedure{ExtractSlice}{$xs$, $t$}
			\Comment $xs$ is a list of $n$ CTs, $t\in[0,R)$
			\State mask $\gets$ [1]*$C$+[0]*($32768-C$)
			\State $x\gets 0$
			\For{$i$ in range$(n)$}
				\State $x\gets x + (((xs_i\cdot (mask \gg C\cdot t)) \ll C\cdot t) \gg C\cdot i)$
			\EndFor
			\State \Return $x$
			\EndProcedure
			\State
			
			\Procedure{Compactify}{$us$, $ps$}
			\State accs $\gets$ [0]*$R$
			\For{$t$ in range$(R)$}
			\Comment Loop over each slice
				\State $u\gets \mathrm{ExtractSlice}(us, t)$
				\State $v \gets \mathrm{RunningSum}(u)$
				\State $xs \gets $[0]*$r$
				\For{$i$ in range$(1, r+1)$}
					\State $w_i \gets \mathrm{map}((==i), v)$
					\Comment 1 where the running-sum is $i$, 0 elsewhere
					\State $xs_i \gets w_i\cdot u$
					\Comment Extracts the $i\th$ 1 from $u$
				\EndFor
				\For{$i$ in range$(k)$}
				\Comment $k$ is the number of slots per value
					\State $p\gets \mathrm{ExtractSlice}(ps_i, t)$
					\For{$j$ in range$(1,r+1)$}
						\Comment We assume that there are at most $r$ 1s in $u$
					    \State $x \gets p\cdot xs_j$
					    \Comment The value corresponding to the $j\th$ 1 from $u$
			  		    \State $y \gets \mathrm{Replicate}(x)$
			  		    \Comment The value is now in all slots
			  		    \State $z \gets y\cdot \mathrm{Indicator}(j\cdot k + i-1)$
					    \Comment The value is now in the appropriate output slot
					    \State accs[t] $\gets$ accs[t]$+z$
			    	\EndFor
				\EndFor
			\EndFor
			\State \Return accs
			\EndProcedure
		\end{algorithmic}
	\end{algorithm}
	
	\section{Failure Modes}
	\subsection{Broken Promise}
	If there are in fact $>M$ matches, this does not necessarily invalidate the results. The effect is more indirect in that it increases the probability that more than $r$ keys match in a given slice.
	
	\subsection{Unlucky Distribution}
	The algorithm returns up to $r$ results per slice, and therefore up to $r\cdot R$ results total. If more than $r$ keys match for a given slice (either because we got \emph{very} unlucky, or because there are more than $M$ matches and we are only slightly unlucky), only up to $r$ results are returned per slice.
	
	\subsection{High-Precision Threshold} 
	One problem that Shai identified with the thresholding step is that the polynomial approximation will return values $\approx 1$ if the dot product is larger than $\tau$ and $\approx 0$ if the dot product is smaller than $\tau$. However, for values very close to $\tau$, the polynomial approximation will return values near $1/2$, which results in inaccurate results. One way to prevent this is to require a ``promise'' that the dot products are sufficiently far away from $\tau$. \enote{Is it possible to use  scheme-switching to do a bit-by-bit comparison?}
\end{document}
